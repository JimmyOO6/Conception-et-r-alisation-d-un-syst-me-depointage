\begin{thebibliography}{2}
\fancyhead[R]{\textit{Bibliographie}}
\addcontentsline{toc}{part}{Bibliographie}
		\bibitem[1]{1} \textsc{IONOS}, \emph{Qu’est-ce qu’une application Web ? Définition et exemples}, 07 mars 2019, Disponible sur <https://www.ionos.fr>, (consulté le 2 juin 2020). 
		
		
        \bibitem[2]{2}  \emph{The benefits of web-based applications}, Disponible sur  <https://www.magicwebsolutions.co.uk>, (consulté le 2 juin 2020).
        
        \bibitem[3]{3} \textsc{Fabien Parrain},  \emph{Capteur intégré tactile d'empreintes digitales à microstructures piezorésistives 
        } 2002/12/02, (consulté le 3 juin 2020).
        
        \bibitem[4]{4}  \textsc{Bhuvan Unhelkar}, \emph{Software Engineering with UML}, 1$^e$ Auerbach Publications, CRC PRESS, 2018.
        
        \bibitem[5]{5} P. \textsc{Roques}, \emph{UML 2 Modéliser une Application Web}, 4$^e$ édition, Eyrolles, Paris, 2008.
        
        \bibitem[6]{6}  \textsc{Pascal Roques & Frank Valléé}, \emph{UML 2 en action de l'analyse des besoins à la conception }, 4$^e$ Eyrolles,2007.
        
        \bibitem[7]{7}  \emph{}, [consulté le 19 juin 2020], Disponible sur : https://laurent-audibert.developpez.com/Cours-UML/?page=mise-en-oeuvre-uml#L9-3-1.
        
        \bibitem[8]{8}  \textsc{Bouchlaghem S, Cherifi F, Khanouche F, Maouche L, Zebboudj s}, \emph{Application Web JAVA EE pour la gestion d'un laboratoire de recherche scientifique}, licence en informatique générale,Béjaia,Univérsité A/Mira,2014,200p.
        
        \bibitem[9]{9} \textsc{Joseph Gabay}, \textsc{David Gabay}, \emph{UML 2 Analyse et conception},  Dunod, Paris, 2008.
        
        \bibitem[10]{10} \textsc{Bernhard Rumpe}, \emph{Agile Modeling with UML},  Springer, Aachen, Germany, 2012.
        
        \bibitem[11]{11} \textsc{Christian Soutou}, \textsc{Frédéric Brouard}, \emph{UML 2 pour les bases de données}, 2$^e$ édition,  EYROLLES, Paris, France.

        \bibitem[12]{12} \textsc{Stéphane Crozat}, \emph{Modélisation avancée en UML et en relationnel},  25 janvier 2018.   
        
        \bibitem[13]{13} \textsc{StackOverFlow}, \emph{Résultats du sondage auprès des développeurs 2019}, Disponible sur : <insights.stackoverflow.com/survey/2019>,(consulter le 4 mars 2020). 

        \bibitem[14]{14} \textsc{world wide web}, \emph{HTML 5.2}, Disponible sur <https://www.w3.org/TR/2017/REC-html52-20171214/introduction.html#a-quick-introduction-to-html>, (consulter le 6 mars 2020).
        
        \bibitem[15]{15} \textsc{world wide web}, \emph{WHAT IS CSS?}, Disponible sur < https://www.w3.org/Style/CSS/>
        
        \bibitem[16]{16}  \emph{JavaScript}, Disponible sur <https://developer.mozilla.org/fr/docs/Web/JavaScript>, (con\\-sulté le 6 mars 2020).
        
        \bibitem[17]{17}  \emph{The Python Tutorial}, Disponible sur <https://docs.python.org/3.9/tutorial/>, (consulté le 6 mars 2020).
        
        \bibitem[18]{18}  \emph{Django}, Disponible sur <https://www.djangoproject.com>, (consulter le 6 mars 2020). 
        
        \bibitem[19]{19}  \emph{Model}, Disponible sur <https://docs.djangoproject.com/fr/3.1/topics/
        db/models/>, (con\\-sulté le 6 mars 2020).
        
        \bibitem[20]{20}  \emph{View}, Disponible sur <https://docs.djangoproject.com/fr/3.1/topics/
        http/views/>,(con\\-sulter le 6 mars 2020).
        
        \bibitem[21]{21} \emph{Template}, Disponible sur <https://docs.djangoproject.com/fr/3.1/
        topics/templates/>, (consulter le 6 mars 2020).
        
        \bibitem[22]{22} \emph{Bootstrap}, Disponible sur <https://getbootstrap.com>, (consultr le 6 mars 2020).
        
        \bibitem[23]{23} \emph{JQuery}, Disponible sur <https://jquery.com>, (consultr le 6 mars 2020).
        
        \bibitem[24]{24} \emph{Crispy-form}, Disponible sur <https://django-crispy-
        forms.readthedocs.io/en/latest/>, (con\\-sulté le 6 mars 2020).
        
        \bibitem[25]{25} \emph{Pillow}, Disponible sur <https://pillow.readthedocs.io/en/stable/>, (consultr le 6 mars 2020).
        
        \bibitem[26]{26} \emph{Django import/export}, Disponible sur <https://django-import-
        export.readthedocs.io/en\\/latest/>, (consulter le 6 mars 2020). 
        
        \bibitem[27]{27} \emph{Popper}, Disponible sur <https://popper.js.org>, (consulter le 6 mars 2020).
        
        \bibitem[28]{28} \emph{Font Awesome}, Disponible sur <https://fontawesome.com>, (consulter le 6 mars 2020).
        
        \bibitem[29]{29} \emph{Chart.js}, Disponible sur <https://www.chartjs.org>, (consulter le 6 mars 2020).
    
        \bibitem[30]{30} \textsc{Pierrick Arlot}, \emph {Les microcontrôleurs STM32 de ST sautent à pieds joints dans l’univers Arduino }[archive],Disponible sur <http://archive.wikiwix.com/cache/index2.php?url=https\%3A\%2F\%2Fwww.lembarque.com\%2Fles-microcontroleurs-stm32-de-st-sautent-a-pieds-joints-dans-lunivers-arduino\_004896>(Consulté le 08/07/2020)

        \bibitem[31]{31} \textsc{Fabien Danieau},\emph{Programmation Arduino en ligne de commande}, Disponsible sur: \\<http://www.francoistessier.info/blog/2011/07/06/programmation-arduino-en-ligne-de-com\\mande/>, (Consulté le 13/07/2020).
        \bibitem[32]{32}  \textsc{Jean-Luc Aufranc}, \emph{Espressif Systems ESP32 Gets Bluetooth LE 5.0/5.1 Certifications}\\ CNX-Software, 03/01/2020, Disponible sur <https://www.cnx-software.com/2020/01/03/\\espressif-systems-esp32-now-supports-bluetooth-le-5-0-5-1/>, (Consulté le 13/07/2020).
       
        \bibitem[33]{33} \emph{ESP32 Datasheet}[archive], Espressif Systems, Disponible sur <https://www.espressif.com\\/sites/default/files/documentation/esp32\texttt{_datasheet_}en.pdf> (Consulté le  23/07/2020).
        
        \bibitem[34]{34} \emph{DY50 Datasheet}, Adafruit Optical Fingerprint Sensor, Disponible sur <https://cdn-learn.adafruit.com/downloads/pdf/adafruit-optical-fingerprint-sensor.pdf> (consulté le 10/07/2020).
     

        \bibitem[35]{35} \textsc{Brian Kernighan}, \textsc{Dennis Ritchie}(trad.Thierry Buffenoir, \emph{Le langage C [« The C Programming Language »]}, Masson, Paris, 1983, 1er éd, 218 p.
        
        \bibitem[36]{36} \textsc{ladyada}, \emph{AdafruitFingerprint Library Documentation}, Release 1.0, Disponible sur \\<https://readthedocs.org/projects/adafruit-circuitpython-fingerprint/downloads/pdf/latest/> (consulté le 15/07/2020).
        
        \bibitem[37]{37} 
        \emph{Slack}, Disponible sur <https://slack.com/intl/fr-dz/features> (consulter le 12/04/2020).
        
        \bibitem[38]{38} 
        \emph{Trello}, Disponible sur <https://trello.com/fr> (consulté le 12/04/2020).
        
        \bibitem[40]{40} 
        \emph{Github}, Disponible sur <https://github.com> (consulté le 12/04/2020).
        
        \bibitem[41]{41} 
        \emph{Discord}, Disponible sur <https://discord.com> (consulté le 12/04/2020).
        
        \bibitem[42]{42} 
        \emph{Visual Code}, Disponible sur <https://code.visualstudio.com> (consulté le 12/04/2020).
        
        \bibitem[43]{43} 
        \emph{Git}, Disponible sur <https://git-scm.com> (consulté le 12/04/2020).
        
        \bibitem[44]{44} 
        \emph{Conda}, Disponible sur <https://docs.conda.io/en/latest/> (consulté le 12/04/2020).
        
        \bibitem[45]{45}
        \emph{AdobeXD}, Disponible sur <https://www.adobe.com/fr/products/xd.html> (consulté le 12/04/2020).
        
        \bibitem[46]{46} 
        \emph{SASS}, Disponible sur <https://sass-lang.com> (consulté le 12/04/2020).
        
        \bibitem[47]{47} 
        \emph{MySql}, Disponible sur <https://www.mysql.com/fr/> (consulté le 12/04/2020).
        
        \bibitem[48]{48} 
        \emph{ORM}, Disponible sur <https://docs.djangoproject.com/fr/3.1/topics/db/queries/> (consulté le 12/04/2020).

		
\end{thebibliography}