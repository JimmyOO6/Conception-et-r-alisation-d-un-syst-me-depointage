\chapter{Contexte du projet et méthodologie de
conception}
\fancyhead[R]{\textit{Contexte du projet et Méthodologie de
conception}}
\renewcommand{\headrulewidth}{1pt}


\section{Introduction}
    Dans ce premier chapitre, nous allons exposer le contexte et la problématique à résoudre du projet. Ensuite, nous aborderons quelques définitions sur les applications web et les empreintes digitales. Pour enfin définir le processus de développement afin de faciliter l’élaboration du projet.


\section{Contexte du projet}

L’entreprise est une organisation qui mobilise des ressources dans le but de produire ou fournir un service, dans un souci vital de rentabilité. Or, le climat économique actuel se distingue par des marqueurs qui rendent la survie des entreprises difficiles. Parmi ces derniers, on peut citer la forte concurrence, l’extrême évolution des marchés et leurs imprévisibilités ainsi que la mondialisation du secteur économique. 

Toute entreprise voulant être prospère se doit de garder les coûts au minimum et les profils au maximum, tout en ayant une administration qui veille à son bon fonctionnement, cependant certaines taches sont répétitives et chronophages, mais ne peuvent pas être négligées, ce qui pousse l’entreprise à déployer des ressources humaines et matérielles considérables dans ses tâches. Ce qui ne représente pas la valeur ajoutée réelle que génère l’entreprise pour son environnement dans son domaine d’expertise. 

Dans l’optique de minimiser les dépenses et de mieux utiliser leurs moyens, les entreprises ont eu recours aux technologies de l’information et de la communication ainsi qu’au système d’information. Du fait de simplement vouloir garder les informations des employés dans une base de données pour y accéder plus facilement, jusqu’à l’utilisation des algorithmes d’intelligence artificielle des big data pour l’aide à la décision. Du simple employé au PDG, tous ont recours aux nouvelles technologies pour mieux accomplir leurs tâches et être plus efficaces et efficients. Parmi ces tâches, nous avons choisi de traiter la gestion de pointage des employés ainsi que leurs temps de travail.  


\section{Problématique}
Le contexte du projet étant établi, dans cette section nous allons décrire la problématique de notre projet à fin de poser les conditions-cadres ainsi que les attentes de ce dernier.

Le but étant de concevoir et de réaliser une application web qui permet une gestion précise du pointage des employés, de leurs temps de travail au sein d’une PME, grâce à une pointeuse biométrique que nous allons réaliser qui sera capable d’identifier de manière unique un individu déjà enregistré et de communiquer avec l’application web.

Nous espérons une fois ce projet à terme, poussez les entreprises à abandonner leurs anciennes méthodes de pointage et de gestion des plannings pour gagner en efficacité et réduire les ressources allouées à ces tâches. En offrant un outil de supervision simple et ergonomique et en collectant les informations qui sont primordiales pour faciliter l’utilisation aux responsables, ainsi qu’un espace individuel dédié à chaque employé dans le but d’avoir son planning et ces informations de pointage de manière transparente.


\section{Les applications web}
    \subsection{Définitions}
        Une application web (ou web App) est un logiciel applicatif hébergé sur un serveur et accessible depuis un navigateur web (Google Chrome, Mozilla Firefox, Safari…). Contrairement à une application native, aucune installation n’est nécessaire ouvrant la porte à de nombreux avantages.
        
    \subsection{Quelle est la différence avec les applications web et les applications natives?}
        Une application web fonctionne généralement comme une application native installée sur votre machine à la différence que celle-ci s’exécute directement sur le navigateur web, ce qui lui permet d’être disponible partout avec des données synchronisées, le tableau \ref{tab1}  ci-dessous présente une comparaison entre les deux types d’application \cite{1} :
        
        \begin{table}[!h]
  \small
  \centering
  \footnotesize{
   \begin{tabular}{|p{4cm}|p{4cm}|p{4cm}|} %%% La taille des trois colonnes est égale à 4cm %%%
    \hline
      & \textbf{Application native} & \textbf{Application web} \\
    \hline
   \textbf{Plateforme} & Dépendant de la plateforme utilisée & Indépendant de la plateforme utilisée \\
    \hline
    \textbf{Stockage de données} & Sur l’appareil de l’utilisateur/ ou sur un sevreur & En général sur le serveur \\
    \hline
    \textbf{Utilisation des fonctions de l’appareil de l’utilisateur} & Totale & Limitée \\
    \hline
    \textbf{Source} & Téléchargement via le fournisseur & Directement sur le navigateur \\
    \hline
    \textbf{Installation} & Nécessaire. & Pas nécessaire. \\
    \hline
    \textbf{Mise à jour} & Doit être téléchargée puis installée & Est intégrée par les fournisseurs et disponible immédiatement après le déploiement \\
    \hline
    \textbf{Connexion internet} & Pas nécessaire la plupart du temps & La plupart du temps nécessaire \\
    \hline
                \end{tabular}}
                \caption{Tableau comparatif entre les applications web et native.} 
                \label{tab1}
        \end{table}
    
    \subsection{Pourquoi une application web ?}
        Les applications web ont considérablement évolué au cours des dernières années avec des améliorations en matière de sécurité avec des technologies de plus en plus flexibles, ce qui permet de développer presque toutes les applications natives en tant qu’application web et de bénéficier des nombreux avantages offerts par le web :
        
        \begin{itemize}

            \item[\textbullet] \textbf{Accessibilité optimisée :}
                Les applications web n’ont pas besoin d’être installées, cela permet un accès universel depuis n’importe quel type de poste.
            
            \item[\textbullet] \textbf{Développement rentable :}
                Il n'est pas nécessaire de programmer et tester sur toutes les versions et configurations des systèmes d'exploitation possible, cela rend le développement moins coûteux et réduit les délais.
            
             \item[\textbullet] \textbf{Installation et maintenance simplifiées :}
                Avec l'approche basée sur le web, l'installation et la maintenance deviennent également moins compliquées. Une fois qu'une nouvelle version ou mise à niveau est installée sur le serveur, elle sera accessible sur n'importe quel type de poste.
            
            \item[\textbullet] \textbf{Technologies de base flexibles :}
                Chacune des technologies de base peut être utilisée pour créer des applications web, en fonction des exigences de l'application.\cite{2}
            
        
        \end{itemize}
        
   
\section{Empreinte digitale}
        \subsection{Caractéristique d’une empreinte digitale}
            Une empreinte digitale se compose d’un ensemble de stries (ici définies comme étant les reliefs positifs qui rentrent en contact avec la surface du capteur) et de sillons définissant le relief de la surface du doigt. 
            
            Les caractéristiques topologiques de l’empreinte restent constantes tout au long de la vie d’un individu et ne peuvent être que partiellement altérées par de profondes coupures laissant apparaître des cicatrices. Le caractère permanent de l’empreinte digitale permet ainsi d’extraire une signature mathématique donnant la possibilité de l’identifier de manière extrêmement fiable.\cite{3}
        
        \subsection{Pourquoi l’empreinte digitale ?}
            Dans un monde en constante évolution en matière d’innovation technologique, la biométrie s’est rapidement distinguée comme la plus pertinente pour identifier et authentifier les personnes de manière fiable et rapide, en fonction de leurs caractéristiques biologiques uniques.

            Dans la perspective de réaliser un système de pointage fiable pour mesurer le temps de travail et gérer la présence des employés, ainsi que de maximiser et motiver la productivité de ces derniers en minimisant les pertes de l’entreprise, il est nécessaire d’avoir un bon système de reconnaissance fiable et simple d’utilisation. Ainsi nous considérons que l’empreinte biométrique est la solution la plus intéressante, tant du point de vue technique et économique. 

\section{Processus du développement}
Un processus définit une séquence d’étapes, partiellement ordonnées, qui concourent à l’obtention d’un système logiciel ou à l’évolution d’un système existant. L’objet d’un processus de développement est de produire des logiciels de qualité qui répondent aux besoins de leurs utilisateurs dans des temps et des coûts prévisibles. \cite{5}

Après avoir analysé de manière globale notre projet, nous avons décidé de travailler selon le processus de développement proposé dans le livre qui s’intitule « UML 2 modéliser une application web » de Pascal Roques. Un processus que décrit l’auteur à mi-chemin entre UP (Unified Process) et les méthodes agiles telles que XP et Scrum, qui s’inspire également des bonnes pratiques prônées par les tenants de la modélisation agile.
    \subsection{Processus unifié (UP)}
        Le processus unifié est un processus de développement logiciel « itératif et incrémental, centré sur l’architecture, conduit par les cas d’utilisation et piloté par les risques » :
    
    \begin{itemize}
        \item [\textbullet] \textbf{Itératif et incrémental :}
            le projet est découpé en itérations de courte durée (environ 1 mois) qui aident à mieux suivre l’avancement global. À la fin de chaque itération, une partie exécutable du système final est produite, de façon incrémentale.
        \item [\textbullet] \textbf{Centré sur l’architecture :}
            tout système complexe doit être décomposé en parties modulaires afin de garantir une maintenance et une évolution facilitées. Cette architecture (fonctionnelle, logique, matérielle, etc.) doit être modélisée en UML et pas seulement documentée en texte.
        \item [\textbullet] \textbf{Piloté par les risques :}
            les risques majeurs du projet doivent être identifiés au plus tôt, mais surtout levés le plus rapidement possible. Les mesures à prendre dans ce cadre déterminent l’ordre des itérations.
        \item [\textbullet] \textbf{Conduit par les cas d’utilisation :}
            le projet est mené en tenant compte des besoins et des exigences des utilisateurs. Les cas d’utilisation du futur système sont identifiés, décrits avec précision et priorisés.
            \cite{5}
    \end{itemize}
    
    \subsection{Les méthodes agiles}
        La notion de méthode agile est née à travers un manifeste signé en 2001 par 17 personnalités du développement logiciel parmi eux Ward Cunningham, Alistair Cockburn, Kent Beck, Martin Fowler, Ron Jeffries, Steve Mellor, Robert C. Martin, Ken Schwaber, Jeff Sutherland, etc. Ce manifeste prône quatre valeurs fondamentales :
    \begin{itemize}
        \item [\textbullet] \textbf{« Personnes et interactions plutôt que processus et outils » :}
            dans l’optique agile, l’équipe est bien plus importante que les moyens matériels ou les procédures. Il est préférable d’avoir une équipe soudée et qui communique, composé de développeurs moyens, plutôt qu’une équipe composée d’individualistes, même brillants. La communication est une notion fondamentale.
        \item [\textbullet] \textbf{« Logiciel fonctionnel plutôt que documentation complète » :}
            il est vital que l’application fonctionne. Le reste, et notamment la documentation technique, est secondaire, même si une documentation succincte et précise est utile comme moyen de communication. La documentation représente une charge de travail importante et peut être néfaste si elle n’est pas à jour. Il est préférable de commenter abondamment le code lui-même, et surtout de transférer les compétences au sein de l’équipe (on en revient à l’importance de la communication).
        \item [\textbullet] \textbf{« Collaboration avec le client plutôt que négociation de contrat » :}
            le client doit être impliqué dans le développement. On ne peut se contenter de négocier un contrat au début du projet, puis de négliger les demandes du client. Le client doit collaborer avec l’équipe et fournir un feedback continu sur l’adaptation du logiciel à ses attentes.
        \item [\textbullet] \textbf{« Réagir au changement plutôt que suivre un plan » :}
            la planification initiale et la structure du logiciel doivent être flexibles afin de permettre l’évolution de la demande du client tout au long du projet. Les premières releases du logiciel vont souvent provoquer des demandes d’évolution\cite{5}. 
    \end{itemize}
    
    \subsection{Le processus choisi }
        Après une brève présentation des concepts majeurs dont s’inspire le processus choisi, nous allons le décrire d’une manière plus détaillée et citer d’où viens chaque caractéristique.
        \begin{itemize}
        
            \item [\textbullet] Un processus conduit par cas d’utilisation, comme UP.
            \item [\textbullet] Relativement léger et restreint, comme les méthodes agiles néanmoins sans négliger les activités de modélisations en analyse et conception.
            \item [\textbullet] Utilise un sous-ensemble nécessaire et suffisant du langage UML conformément à AM.
            \item [\textbullet] Veille à modéliser tous les aspects critiques du système.
            
        \end{itemize}
        Les besoins sont modélisés en cas d’utilisation UML pour être représentés de façon plus concrète par des maquettes IHM, dans le but de les présenter aux futurs utilisateurs. Puis nous allons produire des digrammes de séquence système pour décrire le système comme une boite noire toute en représentant graphiquement la chronologie des interactions entre les acteurs et le système dans le cadre d’un scénario nominal. Grâce aux diagrammes de cas d’utilisation ainsi qu’aux maquettes on pourra modéliser les diagrammes de classes participantes qui décriront les cas d’utilisation, en ayant recours aux trois principales classes d’analyse les classes dialogues, contrôles, entités ainsi que leurs relations.

        À la suite de cela, nous modéliserons les différents diagrammes d’interactions ou chaque cas d’utilisations est décrit en détail dans le but de mettre en évidence l’allocation de responsabilités de chaque objet intervenant dans le cas d’utilisations traiter dans les différents scénarios possibles (nominale/erreur). Enfin, nous pourrons définir les diagrammes de classes de conception qui représente la structure statique du code par le biais des attributs et des relations entre classes ainsi que des opérations décrivant la responsabilité dynamique des classes logicielle.
        La figure \ref{fig1} suivante résume la totalité des diagrammes à modéliser dans le processus choisi.
        
       

    \begin{figure}[h!]
      \centering
      \includegraphics[width=12cm]{images/processus_dev.png}
      \vspace{-10pt}
      \caption{Récapitulatif du processus de développement \cite{5}}
      \label{fig1}
    \end{figure}
\vspace{-30pt}
\section{Conclusion}
    Ce premier chapitre nous a permis de présenter le cadre général du projet, à savoir le contexte et la problématique en élaborant une solution à cette dernière. Nous avons aussi brièvement défini les applications web et les empreintes digitales. Enfin, nous avons relaté la genèse à suivre tout au long du projet.
    

