\renewcommand{\headrulewidth}{0pt}
\fancyhead[R]{}

\begin{titlepage}
\newpage
\pagestyle{fancy}      
\lhead{}  
\chead{}     
\rhead{}     

\renewcommand{\headrulewidth}{0.5pt}

\begin{center}\huge{\textbf{Résumé}} \\ \end{center} 

Ce document a été rédigé en vue de l’obtention du diplôme de master en génie
logiciel. La problématique traitée est la gestion des heures de travail des
employés ainsi que leurs pointages au sein d’une entreprise. À la suite
d’une analyse de la problématique, une solution a été proposée. Composée
d’une application web et d’une pointeuse biométrique, cette alternative
offrira aux PMEs la possibilité de répondre à ce besoin de gestion de
manière plus fiable et plus efficace. Pour concrétiser cela, nous avons eu
recours à un processus de développement et au langage de modélisations UML
pour les phases d’analyse et de conceptions. Pour ce qui est de la phase de
réalisation, elle s’est caractérisée par l’utilisation de Django, qui est un
framwork Python, ainsi que d’autres librairies pour l’implémentation de
l’application web. Quant à la pointeuse, elle est composée du module ESP32
(dans lequel nous avons injecté un programme en C tel que pratiqué sur la
plateforme Arduino) et d’un capteur d’empreinte biométrique DY50.  

\emph{\textbf{Mots clés:} empreinte biométrique , pointage, entreprise, UP, UML, 
    agile, web, Python Django, ESP32}

\vspace{-20pt}
\begin{center}\huge{\textbf{Abstract}} \\ \end{center}

This document was written for a Master’s degree in software engineering. It
centers around the management of working hours and employees clocking within
a company. After this issue was analysed, the realization of a timekeeping
and working hours management system was proposed. This system, consisting of
a web application and a biometric timekeeper, could fulfil SMEs’ needs for a
better and more reliable management systems. For this project, our methodology
consisted of the usage of the Unified Modelling Language (UML) for the analysis
and the design activities. Moreover, the web application was developed using
Python and the Django framework, as well as other libraries. The timekeeping
device, for its part, consists of an ESP32 module (in which we imported a
program coded in C using the Arduino platform) and a DY50 biometric
fingerprint sensor. 

\emph{\textbf{Keywords:}biometric fingerprint, Attendance, business, UP, UML, 
    agile, web, Python Django, ESP32}

\end{titlepage}
