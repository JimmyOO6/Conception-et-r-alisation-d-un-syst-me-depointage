\renewcommand{\headrulewidth}{0pt}
\fancyhead[R]{}

\begin{titlepage}
\newpage
\pagestyle{fancy}      
\lhead{}  
\chead{}     
\rhead{}     
    
\renewcommand{\headrulewidth}{0.5pt}

\begin{center}\huge{\textbf{Résumé}} \\ \end{center} 

Ce document a été rédigé en vue de l’obtention du diplôme de master en génie logiciel. La problématique traitée est la gestion des heures de travail des employés ainsi que leurs pointages au sein d’une entreprise. À la suite d’une analyse de la problématique, une solution a été proposée. En l’occurrence la réalisation d’un système de pointage et de gestion des heures de travail. Composé d’une application web et d’une pointeuse biométrique, cette alternative offrirait aux PME la possibilité de répondre à ce besoin de gestion de manière plus fiable et efficace. Pour concrétiser cela nous avons eu recours à un processus de développement, un langage de modélisations UML pour les activités d’analyse et de conceptions. Par ailleurs, la phase de réalisation s’est caractérisée par l’utilisation de Django qui est un framwork Python ainsi que d’autres librairies pour l’implémentation de l’application web. Pour ce qui est de la pointeuse, elle est composée du module ESP32 (dans lequel nous avons injecté un programme en C grâce à la plateforme Arduino) et d’un capteur d’empreinte biométrique DY50.  

\emph{\textbf{Mots clés:} empreinte biométrique , pointage, entreprise, UP, UML, agile, web, Python Django, ESP32}
\vspace{-20pt}
\begin{center}\huge{\textbf{Abstract}} \\ \end{center}

This document was written for a Master’s degree in software engineering. It centres around the management of working hours and employee clocking within a company. After this issue was analysed, the realization of a timekeeping and working hours management system was proposed. This system, consisting of a web application and a biometric timekeeper, could fulfil SMEs’ needs for a better and more reliable management system.For this project, our methodology consisted of the usage of a Unified Modelling Language (UML) for analysis and conception activities. Moreover, the web application was developed using Python and the Django framework, as well as other libraries. The timekeeping device, for its part, consists of an ESP32 module (in which we imported a program coded in C using the Arduino platform) and a DY50 biometric fingerprint sensor. 
This document was written for a Master’s degree in software engineering. It centres around the management of working hours and employee clocking within a company. After this issue was analysed, the realization of a timekeeping and working hours management system was proposed. This system, consisting of a web application and a biometric timekeeper, could fulfil SMEs’ needs for a better and more reliable management system.For this project, our methodology consisted of the usage of a Unified Modelling Language (UML) for analysis and conception activities. Moreover, the web application was developed using Python and the Django framework, as well as other libraries. The timekeeping device, for its part, consists of an ESP32 module (in which we imported a program coded in C using the Arduino platform) and a DY50 biometric fingerprint sensor.

\emph{\textbf{Keywords:}biometric fingerprint, Attendance, business, UP, UML, agile, web, Python Django, ESP32}

\end{titlepage}