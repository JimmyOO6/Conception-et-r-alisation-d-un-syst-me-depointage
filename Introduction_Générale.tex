\chapter*{Introduction générale}
\fancyhead[R]{\textit{Introduction générale}}
\renewcommand{\headrulewidth}{1pt}
\addcontentsline{toc}{part}{Introduction générale}
\onehalfspacing
\thispagestyle{empty}
    
% Remarque générale: vos phrases sont longues.
Une entreprise est une organisation qui rassemble des moyens matériels ainsi que
des personnes qui mobilisent leurs talents et leurs énergies afin de fournir un
service ou un produit à ses clients. Avec l’avènement des Nouvelles Technologies
de l’information et de la Communication (NTIC), de plus en plus d’entreprises font
appel aux nouvelles technologies pour rester dans l’air du temps et faire face à
la concurrence.

En effet avec leurs anciennes méthodes de gestion, les entreprises sont dans
l’obligation de déployer des moyens humains et financiers pour faire face à des
taches répétitives, mais nécessaires. Dans cette optique, ce projet a pour but
d'offrir un système de
gestion de pointage qui sera composé d’une application Web disposant de
fonctionnalités très étendues destinées à différents types d’employés, et d’une
pointeuse biométrique afin d’automatiser l’une des tâches centrales de toute
entreprise, et ce en gardant l’heure de pointage.

Le premier chapitre introduit le contexte du projet et sa problématique, ainsi
que les concepts clés tels que les applications Web et les empreintes digitales.
Par la suite, nous définirons la méthodologie avec laquelle nous avons organisé
notre projet de façon rationalisée et structurée pour nous aider à accomplir
chaque étape du projet, de la planification à la mise en œuvre de façon
efficace.

Le deuxième chapitre qui s’intitule spécification et analyse des besoins,
permettra d’identifier les différents acteurs et leurs cas d’utilisation
respectifs afin de modéliser leurs diagrammes et de les décrire de façon
détaillée ainsi qu’exprimer les exigences fonctionnelles et non fonctionnelles.
Nous modéliserons, ensuite, les diagrammes de séquence système pour enfin
réaliser le prototype de l’application Web.

Le troisième chapitre a pour but de détailler la phase de conception de notre
système. Nous élaborerons d’abord les modèles du domaine, puis les diagrammes de
classes participantes afin de détailler les diagrammes de séquence système
réaliser dans le chapitre précèdent. Ensuite, nous passerons aux diagrammes de
classes de conception préliminaire qui nous permettront de réaliser le diagramme de
classe de conception. Pour enfin
implémenter notre base de données à l’aide du modèle relationnel. 

% suivant peut induire en érreur. car le suivant de celui en cours c'est le
% chap1.
Dans le quatrième chapitre, nous aborderons l’aspect matériel et logiciel qui
nous permettra de réaliser la pointeuse biométrique, puis nous présenterons le
prototype et le fonctionnement de cette dernière.
\clearpage

Quant au dernier chapitre, il expose l’environnement de travail qui est composé
des plateformes, logiciels, et technologies utilisés pour la réalisation de
l’application Web. Nous clôturerons ce chapitre par la présentation de
cette dernière.

Enfin, nous conclurons ce travail en résumant les connaissances acquises durant la réalisation de
ce projet et nous dégagerons quelques perspectives.
