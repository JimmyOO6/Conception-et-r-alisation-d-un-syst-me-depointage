\chapter*{Conclusion générale et perspectives}
\fancyhead[R]{\textit{Conclusion générale et perspectives}}
\addcontentsline{toc}{part}{Conclusion générale et perspectives}

Ce travail a été réalisé dans le cadre de notre projet de fin de cycle master en
Génie logiciel. Il consisté en l'élaboration d'un système de pointage à
empreinte barométrique constitué d'une application web et d'une pointeuse
barométrique. Ce système est destiné aux PME voulant avoir un système de gestion
de pointage automatisé afin de réduire le coût et le taux d'erreurs liés a la
gestion des heures de travaille des employés.

Grâce à un processus de développement, nous avons pu effectuer une spécification
et une analyse des besoins des utilisateurs, et dégager les principaux
acteurs et cas d'utilisation pour réaliser les différents diagrammes, à savoir
les diagrammes de cas d'utilisation et de séquence système. Ensuite, nous avons
conclu cette phase en produisant des maquettes du système à réaliser. La phase
de conception une fois terminée, nous a permis de générer l'ensemble des
diagrammes et de la documentation nécessaire pour commencer la réalisation de
l'application; à savoir les modelés du domaine,  les digrammes de classes
participantes, les diagrammes de séquence, les diagrammes de classes conception,
ainsi que le modelé relationnel. Enfin, nous avons abordé la dernière phase
aisément grâce au travail effectué dans les phases précédentes. Nous avons pu
ainsi utiliser différents outils et plateformes (python, C, Django, MySQL,
Bootstrap, SASS, Arduino, etc.) pour implémenter notre solution que nous avons
dotée d'une identité graphique reflétant le domaine d'activité de celle-ci.

La réalisation de la partie matérielle s'est faite au fur et à mesure de
l'avancement du projet. N’étant pas des professionnels du domaine électronique,
nous avons consacré beaucoup de temps pour la maîtrise et l'apprentissage des
bases de ce domaine pour élaborer une pointeuse biométrique en combinant
différents composants (module ESP32, capteur d'empreinte DY50, écran OLED, etc.)   

\emph{'ATTime'}, est une application web qui permet non seulement au responsable
d'avoir accès aux informations de pointages et de présences des employés en temps
réel, mais elle aussi d'établir des plannings de façon dynamique et de
les affecter aux collaborateurs concernés, de déléguer la supervision d'un
groupe d'employés grâce au rôle de manager, et d'avoir une base de données
contenant toutes les informations pertinentes à propos du personnel en plus de
leurs informations de pointage. Tout ceci en identifiant de manière
unique et infaillible chaque individu lors de son entrée ou sortie de
l'entreprise grâce à la pointeuse.   

Malgré tout le travail fourni, nous sommes conscients que plusieurs aspects de
notre système peuvent et doivent être améliorés. On peut citer à titre d'exemple
l'optimisation du code pour permettre une meilleure exploitation des ressources
matérielles. En guise de perspectives, nous aspirons à enrichir l'application
web avec d'autres fonctionnalités telles que la génération des bulletins de
salaire en fonction des heures de travails, et un système de ticket multicanal.
Concernant la pointeuse biométrique, nous souhaitons pouvoir la rendre
totalement mobile en y intégrant une batterie rechargeable, et un traqueur GPS
dans le but de pouvoir enregistrer la position du pointage en plus de
l'identifiant de l'employé ainsi que son horodatage. Ceci serait spécialement
utile pour les entreprises ayant des employés mobiles ou des lieux de travail
temporaires tels que les chantiers par exemple.  
